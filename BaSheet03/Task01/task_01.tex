\documentclass[a4paper,10pt]{article}
\usepackage[utf8]{inputenc}

%opening
\title{}
\author{}

\usepackage{amsmath,amssymb,amsfonts,amsthm,mathtools} % Mathematik

\usepackage{color}

\begin{document}

\section*{Task 1}

\noindent {\textit{To prove}}:
\begin{align*}
  e^{-r T}\cdot \mathbb{E}\left[ V\left(S,T\right)\right] &=S(0)\cdot A \cdot \Phi\left(d+\sigma \sqrt{T_1}\right)-K e^{-r T} \Phi(d)
\end{align*}
with
\begin{align*}
  A &=e^{-r\left(T-T_2\right)-\sigma^2\left(T_2-T_1\right)/2}\\
  d &=\frac{\log\left(S(0)/K\right)+\left(r-\frac{1}{2}\sigma^2\right)T_2}{\sigma \sqrt{T_1}}\\
  T_1&=T-\frac{M\left(M-1\right)\left(4M+1\right)}{6M^2}\Delta t = \frac{M (1+M) (1+2 M)}{6 M^2}\Delta t\\
  T_2&=T-\frac{\left(M-1\right)}{2}\Delta t=\frac{M+1}{2}\Delta t\\
\end{align*}
{\textit{Proof:}}\\
First transform the payoff function so that it only depends on one variable:
\begin{align*}
 V\left(S,T\right)=& \max \left\{\left(\prod_{i=1}^M S\left(t_i\right)\right)^{1/M}-K,0\right\}\\
  =& \max \left\{\left(\prod_{i=1}^M  S\left(0\right) e^{\left(r-\frac{1}{2}\sigma^2\right)t_i+\sigma W\left(t_i\right)}  \right)^{1/M}-K,0\right\}\\
  =& \max \left\{\left(S\left(0\right)^M e^{\sum_{i=1}^M\left(r-\frac{1}{2}\sigma^2\right)t_i+\sigma W\left(t_i\right)}  \right)^{1/M}-K,0\right\}\\
  =& \max \left\{S\left(0\right)\left( e^{\left(r-\frac{1}{2}\sigma^2\right)\sum_{i=1}^M \Delta t \cdot i+\sigma \sum_{i=1}^M W\left(t_i\right)}  \right)^{1/M}-K,0\right\}\\
  =& \max \left\{S\left(0\right) e^{\frac{1}{M}\left(\left(r-\frac{1}{2}\sigma^2\right) \Delta t \frac{M\left(M+1\right)}{2}+\sigma \sum_{i=1}^M W\left(t_i\right)\right)} -K,0\right\}
% =& \max \left\{\left(\prod_{i=1}^M  S\left(0\right) e^{\left(r-\frac{1}{2}\sigma^2\right)t_i+\sigma W\left(t_i\right)}  \right)^{1/M}-K,0\right\}
\end{align*}
Since $W(t_i)$ are normal distributed and the increments of the Winer process are independent, it holds:
\begin{align*}
 \sum_{i=1}^{M}W\left(t_i\right)=&\sum_{i=1}^{M}\left(M-i+1\right)\left(W\left(t_i\right)-W\left(t_{i-1}\right)\right)\text{ $ $ $ $ $ $ $ $ $ $ $ $ $ $ $ $ with $t_0=0$}\\
 \sim  & \mathcal{N}\left(0, \sum_{i=1}^{M} \left(M-i+1\right)^2 \left(t_i - t_{i-1}\right)\right)\\
 =     & \mathcal{N}\left(0, \Delta t \left(\sum_{i=1}^{M} M^2-2 i M + 2 M + i^2 - 2 i +1\right) \right)\\
 =     & \mathcal{N}\left(0, \Delta t \left( M^3 + 2 M^2 + M - 2 \left(M + 1\right)\sum_{i=1}^{M}i + \sum_{i=1}^{M}i^2 \right) \right)\\
 =     & \mathcal{N}\left(0, \Delta t \left( M^3 + 2 M^2 + M - 2 \left(M + 1\right) \frac{M^2+M}{2} + \frac{(2 M+1) (M + 1) M}{6} \right) \right)\\
 =     & \mathcal{N}\left(0, \Delta t \cdot \frac{1}{6} M(1+M)(1+2M)\right)\\
%  =& \mathcal{N}\left(0, \Delta t\left( M^2+M-\frac{M^2+M}{2}\right) \right)&&=\mathcal{N}\left(0, \Delta t\cdot\frac{M^2+M}{2} \right)\\
\end{align*}
With this we get the following univariate integrand for $\mathbb{E}\left[ V\left(S,T\right)\right]$:
\begin{align*}
 f_{geom}^{disc}\left(s\right):=&\left(S(0) \exp\left(\left(r-\frac{1}{2}\sigma^2\right) \Delta t \cdot\frac{M+1}{2}+\frac{\sigma}{M}\sqrt{\Delta t \cdot \frac{1}{6} M(1+M)(1+2M)}\cdot s\right)-K\right)^+\\
 =&\left(S(0) \exp\left(\left(r-\frac{1}{2}\sigma^2\right) T_2+\sigma\sqrt{T_1}\cdot s\right)-K\right)^+\\
\end{align*}
Now compute:
\begin{align*}
 &f\left(\chi\right)=0\\
 \Leftrightarrow& \left(r-\frac{1}{2}\sigma^2\right) T_2+\sigma\sqrt{T_1} \chi = \log \left(\frac{K}{S(0)}\right)\\
%  \Leftrightarrow& \sigma\sqrt{\Delta t\cdot\frac{M (1+M) (1+2 M)}{6 M^2}} \chi = \log \left(\frac{K}{S(0)}\right)-\left(r-\frac{1}{2}\sigma^2\right)T_1\\
%  \Leftrightarrow& \sigma\sqrt{T_1} \chi = \log \left(\frac{K}{S(0)}\right)-\left(r-\frac{1}{2}\sigma^2\right)T_2\\
 \Leftrightarrow& \chi = \frac{- \log \left(\frac{S(0)}{K}\right)-\left(r-\frac{1}{2}\sigma^2\right)T_2}{\sigma\sqrt{T_1}} = - d\\
\end{align*}
For the expectation we get:
\begin{align*}
 \mathbb{E}\left[ V\left(S,T\right)\right] = & \frac{1}{\sqrt{2\pi}}\int_\infty^\infty f_{geom}^{disc}\left(s\right) \cdot e^{-\frac{1}{2}s^2} ds\\
  = & \frac{1}{\sqrt{2\pi}} \int_\infty^d \left(S(0) \exp\left(\left(r-\frac{1}{2}\sigma^2\right)T_2+\sigma\sqrt{T_1}\cdot s\right)-K\right) \cdot \exp\left(-\frac{1}{2}s^2\right) ds\\
  =& \frac{S(0)}{\sqrt{2\pi}} \exp\left(\left(r-\frac{1}{2}\sigma^2\right)T_2\right) \int_\infty^d \exp\left(-\frac{1}{2}s^2 + \sigma\sqrt{T_1}\cdot s\right) ds\\
   & -\frac{K}{\sqrt{2\pi}} \int_\infty^d e^{-\frac{1}{2}s^2} ds\\
  =& \frac{S(0)}{\sqrt{2\pi}} \exp\left(\left(r-\frac{1}{2}\sigma^2\right)T_2 \right) \int_\infty^d \exp\left( -\frac{1}{2}(s+\sigma\sqrt{T_1})^2+\frac{1}{2}\sigma^2 T_1\right) ds\\
   & -K \cdot \Phi(d)\\
  =& S(0) \exp\left(\left(r-\frac{1}{2}\sigma^2\right)T_2 +\frac{1}{2}\sigma^2 T_1 \right) \Phi\left(\sigma \sqrt{T_1} + d\right) -K \cdot \Phi(d)\\
  =& S(0) \exp\left(r \cdot T_2 -\frac{1}{2}\sigma^2 \left(T_2 - T_1\right) \right) \Phi\left(\sigma \sqrt{T_1} + d\right) -K \cdot \Phi(d)\\
\end{align*}

So the final result is:
\begin{align*}
 V(S,0) = & e^{-r T}\mathbb{E}\left[ V\left(S,T\right)\right]\\
        = & e^{-r T} \left( S(0) e^{r \cdot T_2 -\frac{1}{2}\sigma^2 \left(T_2 - T_1\right) } \Phi\left(\sigma \sqrt{T_1} + d\right) -K \cdot \Phi(d) \right)\\
        = & S(0) e^{-r \cdot (T - T_2) -\frac{1}{2}\sigma^2 \left(T_2 - T_1\right) } \Phi\left(\sigma \sqrt{T_1} + d\right) -K \cdot e^{-r T} \cdot \Phi(d)\\
        = & S(0) \cdot A \cdot \Phi \left(\sigma \sqrt{T_1} + d\right) -K \cdot e^{-r T} \cdot \Phi(d)\\
\end{align*}
\flushright{$\qed$}


\end{document}
